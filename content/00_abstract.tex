\thispagestyle{plain}

\section*{Kurzfassung}
Schwerpunkt dieser Arbeit ist die Gamma/Hadron Separation. 
Dazu wird geprüft, ob diese verbessert werden kann, indem gemessene Untergrunddaten statt Monte Carlo-Simulationen verwendet werden.
Da FACT keine separaten Untergrundmessungen durchführt, werden die Datensätze für diese Arbeit von Quellbeobachtungen so gefiltert, dass annähernd reine Untergrunddatensätze entstehen.
Mit diesen Daten werden ein \texttt{Random Forest} sowie ein \texttt{XGBoost Classifier} trainiert und der AUC-Wert sowie die Signifikanz von Quellen, der auf simulierten und gemessenen Untergrunddaten trainierten Klassifizierer verglichen. 
Die auf gemessenen Daten trainierten Modelle separieren zwar nach dem AUC-Wert besser, weisen jedoch niedrigere Signifikanzen auf. 
Möglicherweise trainieren die Modelle dabei auf den Unterschied zwischen simulierten und gemessenen Daten. 
Diese Methode eignet sich somit nicht die Separation zu verbessern. 
Unter der Annahme, dass die Monte Carlo-Mismatches bei den Simulation verringert werden könnten, würde die Methode voraussichtlich bessere Ergebnisse als die mit Monte Carlo-Protonen trainierten Modelle liefern.

\section*{Abstract}
\begin{english}
The abstract is a short summary of the thesis in English, together with the German summary it has to fit on this page.
\end{english}
