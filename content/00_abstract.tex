\thispagestyle{plain}

\section*{Kurzfassung}
Das First G-APD Cherenkov Telescope (FACT) observiert Quellen hochenergetischer Gammastrahlung.
Um aus den Messungen Informationen über eine Quelle zu bekommen, muss zunächst das Quell-Signal (Gamma-Ereignisse) vom Untergrund (Hadron-Ereignisse) separiert werden.
Dazu werden maschinelle Lerner verwendet, die für beide Klassen (Gamma/Hadron) mit Monte Carlo-Simulationen trainiert werden.
Ziel dieser Arbeit ist es, zu überprüfen, ob die Gamma/Hadron-Seperation verbessert werden kann, indem gemessene Untergrunddaten statt Monte Carlo-Simulationen zum trainieren der Lerner verwendet werden.
Da FACT keine separaten Untergrundmessungen durchführt, werden die Datensätze von Quellbeobachtungen für diese Arbeit so gefiltert, dass annähernd reine Untergrunddatensätze entstehen.
Mit diesen Daten werden ein \texttt{Random Forest} sowie ein \texttt{XGBoost Classifier} trainiert und der ROC-AUC-Wert sowie die Signifikanz für die Detektion von Quellen der auf simulierten und gemessenen Untergrunddaten trainierten Klassifizierer verglichen. 
Obwohl mit den gemessenen Untergruddaten als Trainingsdatensatz ein besserer ROC-AUC-Wert für den Lerner erreicht werden konnte, sank die Signifikanz für die Detektion der Quelle im Verglich zu einem Lerner, der mit simulierten Daten trainiert wurde. 
Möglicherweise trainieren die Modelle den Unterschied zwischen simulierten und gemessenen Daten. 
Diese Methode verbessert derzeitig die Separation nicht. 
Unter der Annahme, dass die Unterschiede zwischen simulierten und gemessenen Daten verringert werden können, hat die Methode allerdings das Potential, die Separation zu verbessern. 
\section*{Abstract}
\begin{english}
The abstract is a short summary of the thesis in English, together with the German summary it has to fit on this page.
\end{english}
