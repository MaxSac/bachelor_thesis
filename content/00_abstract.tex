\thispagestyle{plain}

\section*{Kurzfassung}
Das First G-APD Cherenkov Telescope (FACT) observiert Quellen hochenergetischer Gammastrahlung.
Um aus den Messungen Informationen über eine Quelle zu bekommen, muss zunächst das Quell-Signal (Gamma-Ereignisse) vom Untergrund (Hadron-Ereignisse) separiert werden.
Dazu werden maschinelle Lerner verwendet, die für beide Klassen (Gamma/Hadron) mit Monte Carlo-Simulationen trainiert werden.
Ziel dieser Arbeit ist es, zu überprüfen, ob die Gamma/Hadron-Separation verbessert werden kann, indem gemessene Untergrunddaten statt Monte Carlo-Simulationen zum trainieren der Lerner verwendet werden.
Da FACT keine separaten Untergrundmessungen durchführt, werden die Datensätze von Quellbeobachtungen für diese Arbeit so gefiltert, dass annähernd reine Untergrunddatensätze entstehen.
Mit diesen Daten werden ein \texttt{Random Forest} sowie ein \texttt{XGBoost Classifier} trainiert und der ROC-AUC-Wert sowie die Signifikanz für die Detektion von Quellen der auf simulierten und gemessenen Untergrunddaten trainierten Klassifizierer verglichen. 
Obwohl mit den gemessenen Untergruddaten als Trainingsdatensatz ein besserer ROC-AUC-Wert für den Lerner erreicht werden konnte, sank die Signifikanz für die Detektion der Quelle im Vergleich zu einem Lerner, der mit simulierten Daten trainiert wurde. 
Möglicherweise trainieren die Modelle den Unterschied zwischen simulierten und gemessenen Daten. 
Diese Methode verbessert derzeitig die Separation nicht. 
Unter der Annahme, dass die Unterschiede zwischen simulierten und gemessenen Daten verringert werden können, hat die Methode allerdings das Potential, die Separation zu verbessern. 
\section*{Abstract}
\begin{english}
First G-APD Cherenkov Telescope (FACT) observe highenergy gamma sources.
To get information from sources, signal (gamma-events) has to separate from background (proton-events). 
In training of the machine learning models, Monte Carlo simulated events are used for each class (gamma/hadron).
Idea of the thesis is to achieve better separation by using measured background instead of Monte Carlo simulated for classifier's training.
FACT haven't measured background-data, so a modified source-data with filtered out gammas is used.
Li and Ma significance and ROC-AUC-Score are estimated for a \texttt{Random Forest} and a \texttt{XGBoost Classifier} to compare modells trained with measured and simulated background.
The significance of the ROC-AUC-Score of the measured background modell is against the presumption lower than the simulated modell.
It seems that the modell make separation between simulated and measured events and not with physically differences.
Model currently doesn't improve separation.
If data-Monte Carlo-mismatch are reduce, this model has maybe the option to optimze separation.
\end{english}
