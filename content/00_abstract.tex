\thispagestyle{plain}

\section*{Kurzfassung}
Schwerpunkt dieser Arbeit ist die Gamma Hadron Seperation. 
Dazu wird geprüft ob diese, durch die Verwendung von gemessenen Untergrunddaten anstelle von Simulierten verbessert werden kann. 
Dazu wird mittels eines $\theta^{2}$-Schnittes auf den Quelldaten die FACT aufnimmt, da kein gemessener Untergrunddatensatz exsestiert, der Untergrunddatensatz erstellt.
Anhand dessen werden ein \texttt{Random Forest} sowie ein \texttt{XGBOOST Classifier} trainiert und der AUC-Wert sowie die Signifikanz von Quellen, anhand der auf simulierten und gemessenen Untergrunddaten trainierten Klassifizierer verglichen. 
Die durch gemessenen Daten tranierten Modelle seperieren zwar nach dem AUC-Wert besser, weisen jedoch niedrigere Signifikanzen auf. 
Möglicherweise trainieren die Modelle dabei auf den Unterschied zwischen simulierten und gemessenen Daten. 
Unter den Vorhanden Monte-Carlo Gamma lässt sich, durch diese Methode, die Seperation nicht verbessern. 
\section*{Abstract}
\begin{english}
The abstract is a short summary of the thesis in English, together with the German summary it has to fit on this page.
\end{english}
