\thispagestyle{plain}

\section*{Kurzfassung}
Schwerpunkt dieser Arbeit ist die Gamma/Hadron Seperation. 
Dazu wird geprüft, ob diese verbessert werden kann, indem gemessene Untergrunddaten statt Monte Carlo-Simulationen verwendet werden.
Da FACT keine separaten Untergrundmessungen durchführt, werden die Datensätze für diese Arbeit von Quellbeobachtungen so gefiltert, dass annähernd reine Untergrunddatensätze entstehen.
Mit diesen Daten werden ein \texttt{Random Forest} sowie ein \texttt{XGBOOST Classifier} trainiert und der AUC-Wert sowie die Signifikanz von Quellen, der auf simulierten und gemessenen Untergrunddaten trainierten Klassifizierer verglichen. 
Die auf gemessenen Daten trainierten Modelle separieren zwar nach dem AUC-Wert besser, weisen jedoch niedrigere Signifikanzen auf. 
Möglicherweise trainieren die Modelle dabei auf den Unterschied zwischen simulierten und gemessenen Daten. 

\texttt{Unter den Vorhanden Monte-Carlo Gamma lässt sich, durch diese Methode, die Seperation nicht verbessern.}
\section*{Abstract}
\begin{english}
The abstract is a short summary of the thesis in English, together with the German summary it has to fit on this page.
\end{english}
