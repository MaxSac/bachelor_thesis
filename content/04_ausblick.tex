\chapter{Zusammenfassung und Ausblick}
Das Ziel der Arbeit war, die Gamma/Hadron-Separation zu verbessern, indem Anstelle der Monte Carlo simulierten Protonen, gemessene Protonen zum Trainieren der Klassifizierer verwendet werden. 
Da diese eher dem gemessenen Untergrund ähneln, sollte eine bessere Klassifizierung erfolgen.

Der gemessene Untergrund wurde mittels eines Schnitts auf dem trennstärksten Parameter $\theta$ erstellt um zu vermeiden, dass bei der Erstellung des Datensatzes erneut Monte-Carlo \textbf{Mismatche} eingehen.
Die Wahl des $\theta^{2}$-Schnittes kann nicht trivial gewählt werden. 
Wird der $\theta^{2}$-Schnitt zu groß gewählt, kommt es dazu, dass Detektoreigenschaften signifikant werden. 
Der $\theta^{2}$-Schnitt darf auch nicht zu klein gewählt werden, da sonst noch verhältnismäßig viele Gamma-Ereignisse in dem erstellen Protonen-Datensatz sind.

Desweiteren wurde festgestellt, dass die Monte Carlo simulierten Gamma-Ereignisse besser, von den gemessenen als von den simulierten Protonen, zu trennen sind. 
Eine mögliche Ursache dafür ist, dass die Modelle zwischen Simulation und gemessenen Ereignisse unterscheiden.

Diese Aussage lässt sich dadurch stützen, dass die Klassifizierer, welche auf echten Daten trainiert wurden, niedrigere Signifikanzen, als die auf den Monte Carlo simulierten Protonen aufweisen. 
Dies widerspricht der anfänglichen Vermutung, dass durch echte Daten die Separation verbessert werden kann. 
Zusätzlich weist ein weniger komplexer Baum wesentlich höhere Signifikanzen auf. 
In Verbindung seiner stark begrenzten Tiefe ist er damit möglicherweise nicht in der Lage zwischen simulierten und gemessenen Protonen zu unterscheiden. 

Desweiteren konnte gezeigt werden, dass durch das Entfernen von Attributen, die nicht dem gemessenen Protonen-Untergrund ähneln die Separation verbessert. 
Dies stützt die These, dass der Klassifizierer zwischen simulierten und echten Daten unterscheidet.

Letztendlich konnte die Gamma/Hadron-Separation nicht durch das Verwenden von gemessenen Protonen-Daten anstelle von simulierten, verbessert werden. 
