\chapter{Zusammenfassung und Ausblick}
Das Ziel der Arbeit war, die Gamma/Hadron-Separation zu verbessern, indem anstelle der Monte Carlo simulierten Protonen gemessene Protonen zum Trainieren der Klassifizierer verwendet werden. 
Da diese eher dem gemessenen Untergrund ähneln, sollte eine bessere Klassifizierung erfolgen.

Der gemessene Untergrund wurde mittels eines Schnitts auf dem trennstärksten Parameter $\theta^{2}$ erstellt.
Die Wahl des $\theta^{2}$-Schnitts liegen dabei folgende Überlegungen zugrunde:
Wird $\theta^{2}$ zu groß gewählt, kommt es dazu, dass Detektoreigenschaften signifikant werden und der Lerner einen großen Teil der $\theta^{2}$-Verteilung im Training nicht auswertet. 
$\theta^{2}$ darf auch nicht zu klein gewählt werden, da sonst noch verhältnismäßig viele Gamma-Ereignisse in dem erstellen Protonen-Datensatz sind.

Lerner welche auf simulierten Gamma- sowie Protonen-Events trainiert wurden, weisen niedrigere AUC-Werte auf, als welche die auf simulierte Gamma- und gemessenen Untergrund-Events trainierten wurden.
Eine mögliche Ursache dafür ist, dass die Modelle zwischen Simulation und gemessenen Ereignisse unterscheiden.

Diese Aussage lässt sich dadurch stützen, dass die Klassifizierer, die auf echten Daten trainiert wurden, zu niedrigeren Signifikanzen führen, als Lerner, die auf den Monte Carlo simulierten Daten trainiert wurden. 
Dies widerspricht der anfänglichen Vermutung, dass durch echte Daten die Separation verbessert werden kann. 
Zusätzlich weist ein weniger komplexer Baum wesentlich höhere Signifikanzen auf. 
In Verbindung seiner stark begrenzten Tiefe ist er damit möglicherweise nicht in der Lage, zwischen simulierten und gemessenen Protonen zu unterscheiden. 

Desweiteren konnte gezeigt werden, dass durch das Entfernen von Attributen, die einen hohen Monte Carlo Mismatch aufweisen, die Separation verbessert werden kann. 
Dies stützt die These, dass der Klassifizierer zwischen simulierten und echten Daten unterscheidet.

Letztendlich konnte die Gamma/Hadron-Separation nicht durch die Verwendung von gemessenen Protonen-Daten anstelle von simulierten im Trainingsdatensatz, verbessert werden. 
